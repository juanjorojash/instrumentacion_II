\documentclass[12pt]{article}
\usepackage[margin=1in]{geometry} 
\input{comunes/preamble}
\usepackage{lastpage}
\usepackage{fancyhdr}
\usepackage{csvsimple,booktabs}
\pagestyle{fancy}
\setlength{\headheight}{42pt}

 
\begin{document}
\lhead{Ingeniería Física \\ Escuela de Física \\ Tecnológico de Costa Rica} 
\rhead{Instrumentación I \\ Tarea \#3  \\ Entrega: Semana 8} 
\cfoot{\thepage\ de \pageref{LastPage}}
\setlength{\parindent}{0em}

\noindent\textbf{Problema a resolver}
\newline
\newline
Se desea crear un altímetro a partir de sensor de presión absoluta modelo:\href{https://www.te.com/commerce/DocumentDelivery/DDEController?Action=srchrtrv&DocNm=MS5803-02BA&DocType=Data+Sheet&DocLang=English}{MS5803-02BA} de TE Connectivity, un microcontrolador y una pantalla. Este sensor entrega un dato digital entre 1000 y 120000 que corresponde a 10mbar y 1200mbar respectivamente, es totalmente lineal. 

\vspace{0.5cm}
\noindent\textbf{Requerimientos:}
\begin{itemize}
    \item El sensor debe ser capaz de enviar un dato al controlador cada \SI{5}{\milli\second}
    \item El controlador debe promediar las mediciones de presión recibidas y enviar un dato de altitud cada 5 segundos al usuario
    \item La resolución debe ser menor a 0.04 \si{\milli bar}
    \item El sensor solo será utilizado en altitudes menores a los 2000 msnm
\end{itemize}

\noindent\textbf{Respoda lo siguiente:}

\begin{enumerate}
    \item Según lo visto en clase y la información disponible en la hoja de datos. ¿Que tipo de sensor es este?
    \item Seleccione la relación de sobremuestreo y el rango de operación del sensor en base a los requerimientos e indíquelo en una tabla.
    \item Realice una tabla resumen con las siguientes características del sensor para la relación de sobremuestreo y rango seleccionados
    \begin{itemize}
        \item Rango
        \item Tiempo de respuesta
        \item Resolución
        \item Exactitud
    \end{itemize}
    \item Se tienen un grupo de mediciones del sensor de presión promediadas cada 5 segundos la cuales corresponden a un viaje en automóvil que se realizó desde Cartago centro a Tobosi de Cartago, ida y vuelta. Los datos se pueden descargar \href{https://estudianteccr-my.sharepoint.com/:u:/g/personal/prof_juan_rojas_estudiantec_cr/ES5u-pJPlFNBnMxKUXgPvi0B6mYtfNTDfYd4Du76JagNqA?e=5g6mPB}{aquí}. Investigue como calcular la altitud a partir de estos datos, calcúlela y genere una gráfica que relacione el tiempo en minutos con la altitud. 
\end{enumerate}
\end{document}