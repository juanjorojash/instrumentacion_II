\documentclass[12pt]{article}
\usepackage[margin=1in]{geometry} 
\usepackage{amsmath}
\usepackage{amssymb}
\usepackage{siunitx}
\usepackage{float}
\usepackage{tikz}
\def\checkmark{\tikz\fill[scale=0.4](0,.35) -- (.25,0) -- (1,.7) -- (.25,.15) -- cycle;} 
\usepackage{url}
\usepackage[siunitx,american,RPvoltages]{circuitikz}
\ctikzset{capacitors/scale=0.7}
\ctikzset{diodes/scale=0.7}
\usepackage{tabularx}
\newcolumntype{C}{>{\centering\arraybackslash}X}
\renewcommand\tabularxcolumn[1]{m{#1}}% for vertical centering text in X column
\usepackage{tabu}
\usepackage[spanish,es-tabla,activeacute]{babel}
\usepackage{babelbib}
\usepackage{booktabs}
\usepackage{pgfplots}
\usepackage{hyperref}
\hypersetup{colorlinks = true,
            linkcolor = black,
            urlcolor  = blue,
            citecolor = blue,
            anchorcolor = blue}
\usepgfplotslibrary{units, fillbetween} 
\pgfplotsset{compat=1.16}
\usepackage{bm}
\usetikzlibrary{arrows, arrows.meta, shapes, 3d, perspective, positioning}
\renewcommand{\sin}{\sen} %change from sin to sen
\usepackage{bohr}
\setbohr{distribution-method = quantum,insert-missing = true}
\usepackage{elements}
\usepackage{verbatim}
\usetikzlibrary{mindmap,trees,backgrounds}
 
\definecolor{color_mate}{RGB}{255,255,128}
\definecolor{color_plas}{RGB}{255,128,255}
\definecolor{color_text}{RGB}{128,255,255}
\definecolor{color_petr}{RGB}{255,192,192}
\definecolor{color_made}{RGB}{192,255,192}
\definecolor{color_meta}{RGB}{192,192,255}
\usepackage[edges]{forest}
\usepackage{etoolbox}
\usepackage{schemata}
\newcommand\diagram[2]{\schema{\schemabox{#1}}{\schemabox{#2}}}
\usepackage{lastpage}
\usepackage{fancyhdr}
\usepackage{csvsimple,booktabs}
\pagestyle{fancy}
\setlength{\headheight}{42pt}
\usepackage{caption}
 
\begin{document}
\lhead{Ingeniería Física \\ Escuela de Física \\ Tecnológico de Costa Rica} 
\rhead{Instrumentación II \\ Tarea \#4  \\ Entrega: Semana 7} 
\cfoot{\thepage\ de \pageref{LastPage}}
\setlength{\parindent}{0em}

Usando LabVIEW desarrolle un instrumento virtual que ejecute una maquina de estados en un período de 1 segundo, la maquina de estados debe:

\begin{itemize}
    \item Tener cinco estados definidos como:
        \begin{itemize}
            \item \emph{stand-by}: estado que espera que se presione el botón, pasa a estado \emph{hello}
            \item \emph{hello}: estado que concatena letra por letra de la palabra "hello" en un string que inicialmente esta vació, debe aparecer una nueva letra cada segundo, al finalizar la palabra pasa a estado \emph{space}
            \item \emph{space}: estado que concatena un espacio vacío al string y espera por 5 segundos, pasa al estado \emph{world}
            \item \emph{world}: estado que concatena letra por letra la palabra "world" en el string, al finalizar la palabra pasa a estado \emph{end}
            \item \emph{end}: estado que termina la ejecución del programa.
        \end{itemize}
    \item Mostrar como avanza la concatenación del string
    \item Mostrar como avanza el tiempo en segundos en cada estado
\end{itemize}

Para finalizar: 
\begin{itemize}
    \item Guarde el instrumento virtual \verb+t4.vi+
    \item Incluya el archivo en un archivo comprimido \verb+t4.zip+ y subalo al TecDigital
\end{itemize}

Cualquier entrega tardía se califica en base a 70. 


% \bibliographystyle{IEEEtran}
% \bibliography{ref_tareas}

\end{document}