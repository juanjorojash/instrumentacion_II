\documentclass[12pt]{article}
\usepackage[margin=1in]{geometry} 
\input{comunes/preamble}
\usepackage{lastpage}
\usepackage{fancyhdr}
\usepackage{csvsimple,booktabs}
\pagestyle{fancy}
\setlength{\headheight}{42pt}

 
\begin{document}
\lhead{Ingeniería Física \\ Escuela de Física \\ Tecnológico de Costa Rica} 
\rhead{Instrumentación I \\ Tarea \#4  \\ Entrega: Semana 11} 
\cfoot{\thepage\ de \pageref{LastPage}}
\setlength{\parindent}{0em}

\noindent\textbf{Problema a resolver}
\newline
\newline
Como ingenieros físicos, se encuentran a cargo mantener activa la \href{https://www.nacion.com/economia/politica-economica/video-franklin-chang-de-ad-astra-rocket-explica/871a4a67-9142-45cd-8df2-d6f4e88f6d81/video/}{línea de producción de hidrógeno como forma de almacenamiento de energía.} El proceso utilizado es la \href{https://en.wikipedia.org/wiki/Electrolysis_of_water}{electrólisis de agua}. Para garantizar que el proceso de producción se mantenga activo, será necesario desarrollar un sistema de medición de nivel de agua para el almacenamiento de agua. 

Para ello, se disponen de varios sensores disponibles: 
\begin{itemize}
    \item \href{https://www.pce-iberica.es/hoja-datos/hoja-datos-pce-ulm-10.pdf}{PCE-ULM 10}
    \item \href{https://www.gavasa.com/wp-content/uploads/2020/09/Wika_LF-1.pdf}{WIKA LM 40.0}
    \item \href{http://www.contrall.com.co/pdf-productos/E2KL.pdf}{E2KL}
    \item \href{https://irp-cdn.multiscreensite.com/964ff6b3/files/uploaded/LTM-2_50061_1.2_pt_eu.pdf}{LTM-2}
    \item \href{https://www.binmaster.com/_resources/dyn/files/75755645zd70b61f0/_fn/PT500_SpecSheet_ES631219062.pdf}{PT-500}
     \item \href{https://www.bermad.com/app/uploads/2016/06/ww-750-80-x-spanish.pdf}{Bermad Modelo 750-80-X}
     \item \href{https://www.burkert.com/en/Media/plm/DTS/DS/DS8189-standard-EU-EN.pdf?id=DTS0000000000000001000244852ENE}{Burket Type 8189}
\end{itemize}
 

\vspace{0.5cm}
\noindent\textbf{Requerimientos:}
\begin{itemize}
    \item Al menos deberá medir una altura 3m en el tanque. 
    \item Un tiempo de respuesta menor a 100 \si{\milli s}.
    \item Deberá poderse integrar a sistemas de medición industrial .
\end{itemize}

\noindent\textbf{Responda lo siguiente:}

\begin{enumerate}
    \item Según lo visto en clase y la información disponible en la hoja de datos. ¿Qué tipos de sensores corresponden cada uno?
    \item Genere una tabla que permita comparar las principales características de los sensores. Tome en cuenta los parámetros vistos y consideraciones vistas en clase.
    \item ¿Cuál de los sensores es la mejor opción para la aplicación requerida? ¿Por qué?
    \item Los sensores de medición por radar, son bastante prácticos dado su facilidad de acople a sistemas de producción. Sin embargo, las olas que puede generar un fluido puede generar una frecuencia de Doppler. ¿En qué consiste está frecuencia? ¿Cómo se elimina este efecto en este tipo de sensores?
\end{enumerate}
\end{document}