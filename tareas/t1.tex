\documentclass[12pt]{article}
\usepackage[margin=1in]{geometry} 
\usepackage{amsmath}
\usepackage{amssymb}
\usepackage{siunitx}
\usepackage{float}
\usepackage{tikz}
\def\checkmark{\tikz\fill[scale=0.4](0,.35) -- (.25,0) -- (1,.7) -- (.25,.15) -- cycle;} 
\usepackage{url}
\usepackage[siunitx,american,RPvoltages]{circuitikz}
\ctikzset{capacitors/scale=0.7}
\ctikzset{diodes/scale=0.7}
\usepackage{tabularx}
\newcolumntype{C}{>{\centering\arraybackslash}X}
\renewcommand\tabularxcolumn[1]{m{#1}}% for vertical centering text in X column
\usepackage{tabu}
\usepackage[spanish,es-tabla,activeacute]{babel}
\usepackage{babelbib}
\usepackage{booktabs}
\usepackage{pgfplots}
\usepackage{hyperref}
\hypersetup{colorlinks = true,
            linkcolor = black,
            urlcolor  = blue,
            citecolor = blue,
            anchorcolor = blue}
\usepgfplotslibrary{units, fillbetween} 
\pgfplotsset{compat=1.16}
\usepackage{bm}
\usetikzlibrary{arrows, arrows.meta, shapes, 3d, perspective, positioning}
\renewcommand{\sin}{\sen} %change from sin to sen
\usepackage{bohr}
\setbohr{distribution-method = quantum,insert-missing = true}
\usepackage{elements}
\usepackage{verbatim}
\usetikzlibrary{mindmap,trees,backgrounds}
 
\definecolor{color_mate}{RGB}{255,255,128}
\definecolor{color_plas}{RGB}{255,128,255}
\definecolor{color_text}{RGB}{128,255,255}
\definecolor{color_petr}{RGB}{255,192,192}
\definecolor{color_made}{RGB}{192,255,192}
\definecolor{color_meta}{RGB}{192,192,255}
\usepackage[edges]{forest}
\usepackage{etoolbox}
\usepackage{schemata}
\newcommand\diagram[2]{\schema{\schemabox{#1}}{\schemabox{#2}}}
\usepackage{lastpage}
\usepackage{fancyhdr}
\usepackage{csvsimple,booktabs}
\pagestyle{fancy}
\setlength{\headheight}{42pt}

 
\begin{document}
\lhead{Ingeniería Física \\ Escuela de Física \\ Tecnológico de Costa Rica} 
\rhead{Instrumentación II \\ Tarea \#1  \\ Entrega: Semana 3} 
\cfoot{\thepage\ de \pageref{LastPage}}
\setlength{\parindent}{0em}


\begin{enumerate}
    \item Usando el estándar \href{http://integrated.cc/cse/Instrumentation_Symbols_and_Identification.pdf}{ANSI/ISA-5.1-2009} realice un diagrama que describa de mejor manera el proceso de medición descrito. El diagrama básico de conexión debe ser como el mostrado en la Figura \ref{fig:while}. El resto del diagrama debe realizarlo en base a ANSI/ISA-5.1-2009
    \begin{figure}[H]
        \centering
        \includegraphics[width=10cm]{fig}
        \caption{Medición de corriente y voltaje en un circuito.}
        \label{fig:while}
    \end{figure}
    
    \begin{table}[H]
        \centering
        \caption{Mediciones tomadas a una frecuencia de \SI{1}{\hertz}}
        \vspace{0.5cm}
        \begin{tabular}{ccc}%
        \toprule
        \bfseries tiempo & \bfseries corriente & \bfseries voltaje \\
        {[\si{\second}]} & [\si{\ampere}] & [\si{\volt}]\\
        %\[\si{\second}\] & \[\si{\ampere}\] & \[\si{\volt}\] \\
        \midrule
        \csvreader[
            late after line=\\,
            late after last line=,
            before reading={\catcode`\#=12},
            after reading={\catcode`\#=6}]%
            {data.csv}{1=\time1,2=\current1,3=\voltage1}{\time1 &\current1 & \voltage1}\\
            \bottomrule
        \end{tabular}
        \label{tab:datos}
    \end{table}

    \item Se realizó una medición por segundo durante 20 segundos usando el proceso de medición descrito y se obtuvieron los datos mostrados en la Tabla \ref{tab:datos}. Calcule el valor de $R_sh$ para cada uno de los $n$ puntos de datos y realice una tabla que incluya las tres columnas de la Tabla \ref{tab:datos} y una tercera columna para los valores de $R_{sh}$.
    \item Para el conjunto de valores de $R_{sh}$, calcule lo siguiente e incluya en una tabla:
    \begin{enumerate}
        \item El valor promedio, $\overline{R_{sh}}$ 
        \item La desviación estándar, $\sigma$
        \item El valor promedio de error relativo tomando \SI{1}{\ohm} como valor real
        \item El valor de la incertidumbre estándar, $\sigma_x = \sigma / \sqrt{n}$
        \item La exactitud
        \item La precisión
        \item La repetibilidad
    \end{enumerate}

\end{enumerate}




% \bibliographystyle{IEEEtran}
% \bibliography{ref_tareas}

\end{document}