\documentclass[12pt]{article}
\usepackage[margin=1in]{geometry} 
\input{comunes/preamble}
\usepackage{lastpage}
\usepackage{fancyhdr}
\usepackage{csvsimple,booktabs}
\pagestyle{fancy}
\setlength{\headheight}{42pt}

 
\begin{document}
\lhead{Ingeniería Física \\ Escuela de Física \\ Tecnológico de Costa Rica} 
\rhead{Instrumentación I \\ Tarea \#6  \\ Entrega: Semana 16} 
\cfoot{\thepage\ de \pageref{LastPage}}
\setlength{\parindent}{0em}

\noindent\textbf{Verificación de la cantidad de luminosidad en el escritorio}\\

La 
\href{https://www.osha.gov/}{Administración de Seguridad y Salud Ocupacional (Occupational Safety and Health Administration, OSHA)}, es uno de los principales entes rectores en temas de salud ocupacional. Un aspecto a destacar es la iluminación requerida en los puestos de trabajo de acuerdo a la actividad, que está regulado en el estándar \href{https://www.osha.gov/laws-regs/regulations/standardnumber/1926/1926.56}{1926.56.}\\

Para oficinas de trabajo, la iluminación mínima requerida es 30 $velas$ o $foot-candel$. Dadas las condiciones producto de la pandemia, la cantidad de trabajo desde la casa se ha incrementado, será necesario verificar la cantidad de luz que se presenta en el escritorio o mesa donde regularmente trabajan. 
\\

Para ello:

\begin{itemize}
    \item Instale la aplicación \textit{Arduino Science Journal} en su teléfono móvil
    \item Configure una toma de datos con el sensor de luz.
    \item Antes de arrancar la medición, asegúrese su teléfono está completamente acomodado en la mesa, ya que las orientaciones del teléfono pueden dar datos erróneos.
    \item Realice tres juegos de mediciones con al menos 30s de duración cada una: el primer set corresponde  2 en por la mañana, una con sólo luz natural y encendiendo la luz artificial (lámpara de escritorio o bombilla del cuarto). El segundo set de datos corresponde a mediciones por la tarde, una con sólo luz natural y encendiendo la luz artificial (lámpara de escritorio o bombilla del cuarto). El último set de medición corresponde a una medición por la noche solamente utilizando la luz natural
    \item Exporte los valores registrados en un archivo \emph{.csv}
    \item Usando la librería \emph{pandas} en Python lea los datos del archivo .csv y limpie los datos de forma que pueda realizar el siguiente paso.   
    \item Determine el promedio de luz para cada una de las mediciones, la desviación estándar. Usando como valor base lo determinado por \textit{OSHA}, calcule el error, la precisión y la repetabilidad del sistema.
\end{itemize}

\noindent\textbf{Tips}

\begin{itemize}
    \item Una vela ($fc$) es una unidad de iluminancia que no pertenece al Sistema Internacional de Unidades. El factor de conversión corresponde a: 
    \begin{equation*}
        E_v[lx] = \frac{E_v[fc]}{0.09290304} 
    \end{equation*}
    \item Si la cantidad de iluminancia no cumple con lo especificado por la normativa, se le recomienda que acondicione su lugar de trabajo para proteger su vista.
\end{itemize}


\noindent\textbf{Entregable}

Un solo archivo .zip que contenga lo siguiente
\begin{itemize}
    \item Un archivo \emph{.py} que incluya comentarios aclaratorios de los pasos que se realizaron y tres gráficas. Una por cada set de medición.
    \item Un archivo \emph{.pdf} donde se analicen los datos obtenidos.
    \item Un archivo \emph{.csv} con los datos de los sensores
\end{itemize}

% \bibliographystyle{IEEEtran}
% \bibliography{ref_tareas}

\end{document}